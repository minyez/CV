%%% Local Variables:
%%% mode: latex
%%% End:

\section{\hei 教育背景}

\tlcventry{2015.9}{0}{\href{http://www.pku.edu.cn}{北京大学}~\href{http://www.chem.pku.edu.cn}{化学与分子工程学院}}{博士研究生}{理论与计算化学 (预期2020年毕业). \raggedleft 导师:蒋鸿~研究员}{}{}

\tlcventry{2011.9}{2015.7}{\href{http://www.pku.edu.cn}{北京大学}~\href{http://www.chem.pku.edu.cn}{化学与分子工程学院}}{理学学士}{化学}{GPA: 3.41/4}{}
\vspace{-1em}
\section{\hei 技能}

\cvcomputer{\hei 理论工具}{密度泛函理论, 多体理论}
           {\hei 计算软件}{VASP, \textsc{WIEN2k}, \textsc{Abinit}, \textsc{GPAW}, \textsc{FHI-gap}}
           
\cvcomputer{\hei 编程语言}{Python, Fortran, Bash, MPI并行, C (基础)}
           {\hei 科学库}{Intel{\textregistered} MKL, spglib, FFTW3, \textsc{Phonopy}, \textsc{ASE}}
           
\cvcomputer{\hei 开发工具}{Vim, VS Code, Git, Makefile, Mathematica{\textregistered}}
           {\hei 可视化}{Matplotlib, XmGrace, Adobe{\textregistered} Photoshop}
           
\cvcomputer{\hei 文档写作}{\LaTeX{}, Markdown, Jupyter Notebook}
           {\hei 外语能力}{CET6 (550), JLPT N1 (113)}{}{}


\section{\hei 经历}
%
%% Center labels and use "Since"
%%\tltextstart[base]{\scriptsize}
%%\tltextend[base]{\scriptsize}
%%\tlsince{Since~}
%

\subsection{\hei 科研项目}

\tlcventry{2017.9}{0}{基于LAPW框架的超越密度泛函理论的第一性原理电子结构方法研究与程序开发}{博士课题}{}{}
{
    \begin{itemize}%
        \item 调试/维护/优化课题组自研的多体微扰理论$GW$全电子计算Fortran程序\textsc{FHI-gap};
        \item 推导混合基组下的截断库仑势,实现低维材料体系自能关于模型参数的加速收敛;
        \item 实现ACFDT-RPA电子相关能全电子计算,调试与\textsc{WIEN2k}的接口,测试并行可靠性.
    \end{itemize}
}

\tlcventry{2016.8}{2017.3}{核壳结构Fe@FeP纳米颗粒催化氢产生反应的理论研究}{合作课题}{}{}
{
    \begin{itemize}
        \item 构建Fe@FeP界面表面模型与不同表面位点的氢吸附模型,进行密度泛函的第一性原理计算;
        \item 在考虑零点能与振动熵情况下验证了Fe@FeP阴极反应自由能变接近于零,佐证其实验上优良催化性能.
    \end{itemize}
}

\tlcventry{2015.12}{2018.4}{二硫化亚铁\ce{FeS2}晶相平衡态热稳定性的理论研究}{博士课题}{}{}
{
    \begin{itemize}%
        \item 基于ACFDT-RPA方法,在计算上首次得到与实验值处于同一数量级的\ce{FeS2}黄铁矿相到白铁矿相的转化焓;
        \item 通过构造有效带隙描述符,从电子结构差别上解释了黄铁矿相热稳定性来源及传统密度泛函近似失效的原因.
    \end{itemize}
}

\tlcventry{2017.10}{0}{从头算程序输入输出文件前后处理的Python脚本集\texttt{mykit}}{独立项目}{}{}
{
    \begin{itemize}%
        \item 适用于多种从头算程序,实现关键词映射,允许不同程序输入文件的近似相互转换;
        \item GitHub开源. 支持连续集成、自动化测试与覆盖度计算 (4061行77.2\%).
    \end{itemize}
}
\subsection{\hei 学生工作}

\tlcventry{2015.9}{2017.7}{支部书记}{北京大学化学与分子工程学院2015级研究生党支部}{}{}
{
  \begin{itemize}%
  \item 统筹支部党员发展、策划党团日活动、定期召开支部生活会和党员大会;
  \item 组织党团日获校级三等奖(2016),本人获评学院优秀党员.
  \end{itemize}
}
\tlcventry{2011.9}{2015.7}{班长}{北京大学化学与分子工程学院2011级本科生一班}{}{}
{
%    \begin{itemize}%
%        \item 
%        \item 
%    \end{itemize}
}

%
%\newpage
%
%\tlcventry{2014}{2013}{\href{http://bbs.chinatex.org}{ChinaTeX}}{论坛精英}{}{}
%{
%  \begin{tightitemize}%
%  \item 在论坛分享了多篇 \LaTeX{} 的学习心得。
%  \item 高校 \LaTeX{} 推广者。开发华南师范大学的硕士生学位 \LaTeX{} 模板
%    \href{https://github.com/scnu/scnuthesis}{SCNUThesis},支持在线编写,并被广东工业大学、首都师范大学等高校引用。
%  \end{tightitemize}}

%\subsection{\hei 个人开源项目}
%\subsection{个人开源项目}
%
%\tlcventry{2017}{}{\href{http://github.com/wzpan/dingdang-robot}{dingdang-robot}}{作者}{}{}%
%  {
%\begin{itemize}
%\item \textbf{项目简述}:类似 Amazon Echo 的机器人叮当。能工作在 Raspberry Pi 上
%  的智能对话机器人,目的是提供一个适用于中文环境的开箱即用的中文对话机器人;
%\item \textbf{主要特性}:支持选择离线唤醒SST引擎、在线SST引擎和TTS引擎,支持
%  接入聊天机器人,支持接入微信,无缝联动 HomeAssistant 。支持插件模块化,容易拓展。
%\item \textbf{项目指数}:520 stars,161 forks。QQ 用户群 407 名。技能插件 24 个。
%\end{itemize}}
%
%\tlcventry{2014}{2013}{\href{http://github.com/wzpan/qtevm}{QtEVM}}{作者}{}{}%
%  {
%\begin{itemize}
%\item \textbf{项目简述}:欧拉影像放大技术(Eulerian Video Magnification)的开源
%  实现;
%\item \textbf{主要特性}:首个完整的 C++ 开源实现,能同时放大动作变化和颜色变化;
%\item \textbf{项目指数}:54 stars,29 forks。
%\end{itemize}}
%
%\tlcventry{2013}{}{\href{http://github.com/wzpan/hexo-theme-freemind}{hexo-theme-freemind}}{作者}{}{}%
%{
%\begin{itemize}
%\item \textbf{项目简述}:最受欢迎的 Hexo 博客主题之一;
%\item \textbf{主要特性}:丰富的 tag 插件,多种颜色主题,本地搜索引擎;
%\item \textbf{项目指数}:272 stars,122 forks。
%\end{itemize}}
%
%\tlcventry{2013}{}{\href{http://github.com/wzpan/hexo-generator-search}{hexo-generator-search}}{作者}{}{}%
%  {
%\begin{itemize}
%\item \textbf{项目简述}:一个能生成 Hexo 站点检索数据的插件;
%\item \textbf{主要特性}:同时支持生成 XML 和
%  JSON 两种格式,支持定制生成的内容范围;
%\item \textbf{项目指数}:107 stars,19 forks。被多个主题内置。NPM 月均下载量 2000 次。
%\end{itemize}
%}

\section{\hei 论文发表}
{
%\begin{tightitemize}
%    \item \underline{\bf Zhang, M.-Y.}; Jiang, H. Electronic Band Structure of Cuprous and Silver Halides: an All-Electron $GW$ Study. \textit{In press}. Preprint:  \href{https://arxiv.org/abs/1906.02472}{arXiv:1906.02472 (2019)
%    \item Cui, Z.-H.; Wang, Y.-C.; \underline{\bf Zhang, M.-Y.}; Xu, X.; Jiang, H. Doubly Screened Hybrid Functional: an Accurate First-Principles Approach for Both Narrow- and Wide-Gap Semiconductors. \textit{J. Phys. Chem. Lett.}  {\bf 2018}, \textit{9}, \href{dx.doi.org/10.1021/acs.jpclett.8b00919}{2338-2345}. (IF=7.329)
%    \item \underline{\bf Zhang, M.-Y.}; Cui, Z.-H.; Jiang, H. Relative Stability of \ce{FeS2} Polymorphs with the Random Phase Approximation Approach. \textit{J. Mater. Chem. A} {\bf 2018}, \textit{6}, \href{https://pubs.rsc.org/en/content/articlelanding/2018/ta/c8ta00759d}{6606}. (IF=10.733)
%    \item Li, X.; Liu, W.; \underline{\bf Zhang, M.}. et al. Strong Metal–Phosphide Interactions in Core–Shell Geometry for Enhanced Electrocatalysis. \textit{Nano Lett.} {\bf 2017}, \textit{17}, \href{dx.doi.org/10.1021/acs.nanolett.7b00126}{2057}. (IF=12.080)
%\end{tightitemize}
\setlength{\parskip}{-6pt}
\tldatelabelcventry{2019/9}{2019.9}{第一作者}{预印本}{Electronic Band Structure of Cuprous and Silver Halides: an All-Electron $GW$ Study. \href{https://arxiv.org/abs/1906.02472}{arXiv:1906.02472 (2019)}}{}{}
\setlength{\parskip}{-8pt}
\tldatelabelcventry{2018/10}{2018.10}{第三作者}{IF=7.329}{Doubly Screened Hybrid Functional: an Accurate First-Principles Approach for Both Narrow- and Wide-Gap Semiconductors. \textit{J. Phys. Chem. Lett.}  $\mathbf{2018}$, \textit{9}, \href{dx.doi.org/10.1021/acs.jpclett.8b00919}{2338-2345}}{}{}
\setlength{\parskip}{-8pt}
\tldatelabelcventry{2018/3}{2018.3}{第一作者}{IF=10.733}{Relative Stability of \ce{FeS2} Polymorphs with the Random Phase Approximation Approach. \textit{J. Mater. Chem. A} $\mathbf{2018}$, \textit{6}, \href{https://pubs.rsc.org/en/content/articlelanding/2018/ta/c8ta00759d}{6606}}{}{}
\setlength{\parskip}{-8pt}
%\tldatelabelcventry{2017}{2017.6}{共同一作}{IF=12.080}{Strong Metal–Phosphide Interactions in Core–Shell Geometry for Enhanced Electrocatalysis. \textit{Nano Lett.} {\bf 2017}, \textit{17}, \href{dx.doi.org/10.1021/acs.nanolett.7b00126}{2057}}{}{}
%\begin{tightenum}
%    \item \underline{\hei Zhang, M.-Y.}; Jiang, H. \textit{In press}. Preprint: \href{https://arxiv.org/abs/1906.02472}{arXiv:1906.02472 (2019)}
%    \item Cui, Z.-H.; Wang, Y.-C.; \underline{\bf Zhang, M.-Y.}; Xu, X.; Jiang, H. \textit{J. Phys. Chem. Lett.}  {\bf 2018}, \textit{9}, \href{dx.doi.org/10.1021/acs.jpclett.8b00919}{2338-2345}. (IF=7.329)
%    \item \underline{\hei Zhang, M.-Y.}; Cui, Z.-H.; Jiang, H. \textit{J. Mater. Chem. A} {\bf 2018}, \textit{6}, \href{https://pubs.rsc.org/en/content/articlelanding/2018/ta/c8ta00759d}{6606}. (IF=10.733)
%    \item Li, X.; Liu, W.; \underline{\hei Zhang, M.}. et al. \textit{Nano Lett.} {\bf 2017}, \textit{17}, \href{dx.doi.org/10.1021/acs.nanolett.7b00126}{2057}. (IF=12.080)
%\end{tightenum}
}


\begin{multicols}{2}
\section{\hei 获得奖项}

% Restore normal labels
%\tltext{\scriptsize}

{
\setlength{\parskip}{-5pt}
\tldatelabelcventry{2017/10}{2017.10}{先锋物理化学奖学金}{校级}{}{}{}
    
\setlength{\parskip}{-20pt}

\tldatelabelcventry{2016/10}{2016.10}{校长奖学金, 王世仪奖学金}{校级}{}{}{}

\setlength{\parskip}{-10pt}

\tlcventry{2015/9}{2019/10}{三好学生($\times3$), 三好学生标兵}{校级}{}{}{}

\setlength{\parskip}{2pt}

\tlcventry{2011/9}{2015/7}{五四奖学金, 先锋奖学金}{校级}{}{}{}

}

\section{\hei 其他活动}
{
    \begin{itemize}
        \setlength{\parskip}{3pt}
        \item {\hei 助教} 中级物理化学 (2017、2018春季学期).
        \item {\hei 翻译} 日语访谈和字幕制作;英文科普文章,即将发表于《环球科学》杂志.
        \item {\hei 博客} 搭建个人主页撰写科学计算相关技术博客.
        \item {\hei 跑步} 舒缓压力. 7公里配速4'30.
        \item {\hei 篮球} 担任院队首发,蝉联北大硕博杯院系篮球赛冠军.
    \end{itemize}
}
\end{multicols}
